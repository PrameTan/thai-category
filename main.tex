% !TeX spellcheck = th_TH
\documentclass[a4paper,12pt]{extbook}

%%%%%%%%%%%%%%%%%%%%%%
\usepackage[dvipsnames]{xcolor}
\usepackage[framemethod=TikZ]{mdframed}

\usepackage{fancyhdr}
\usepackage{multicol, array, tabularx, enumitem}
%%%%%%%%%%%%%%%%%%%%%%

\usepackage{tikz-cd}

% Set the locale for linebreak to Thai
\XeTeXlinebreaklocale "th"
% In English, when TeX tries to justify text,
% it will add some spaces between words
% For Thai, we "must not" add any space between words
% i.e. put "zero" space between words
\XeTeXlinebreakskip = 0pt plus 0pt
% For a bit better(?) justified output
\sloppy

\usepackage[top = 1in, bottom = 1in, left=1in, right = 1in]{geometry}

\usepackage{amsthm, amsmath, amssymb, mathtools, indentfirst}


%\usepackage[sb]{libertinus}
\usepackage{lmodern}

% For any unicode characters, require XeTeX/XeLaTeX
\usepackage{fontspec}
\defaultfontfeatures{Mapping=tex-text} 
\setmainfont{Latin Modern Roman}
% Set main fonts
% For Thai, I recommend to scale the size to the uppercase size of latin alphabet
%\setmainfont[Scale=MatchLowercase,Mapping=tex-text]{TH Sarabun New}
%%\setmainfont{TeX Gyre Termes}				% Free Times
%%
%%% Sans font
%%\setsansfont{TeX Gyre Heros}				% Free Helvetica
%%
%%% Monospace font
%%\setmonofont{TeX Gyre Cursor}				% Free Courier
%\SetSymbolFont{operators}{normal}{OT1}{tex-gyre-pagella}{m}{n}
%\DeclareSymbolFont{symbols}{OMS}{cmsy}{m}{n}
%\DeclareSymbolFont{largesymbols}{OMX}{cmex}{m}{n}

%\setmathfont{LibertinusMath-Regular.otf}

%% Because latin font in Sarabun is Sans Serif, we prefer to use Serif font
\newfontfamily{\thaifont}[Scale=MatchLowercase,Mapping=tex-text]{TH Sarabun New}

% Set environment for Thai fonts
\newenvironment{thailang}
{\thaifont}
{}

% For automatic switching between languages
\usepackage[Latin,Thai]{ucharclasses}

% When using Thai characters use thailang environment
\setTransitionTo{Thai}{\begin{thailang}}
% For other characters, switch back to the original environment
\setTransitionFrom{Thai}{\end{thailang}}

% Single spacing is too tight for Thai
\usepackage[nodisplayskipstretch]{setspace}
\onehalfspacing

\usepackage{etoolbox}

% For thaialph numbering \thAlph
\usepackage{polyglossia}          
% Set the normal language to English
% i.e. numbering, latin characters will use English font
\setdefaultlanguage{english}
% When using Thai characters, the font will be automatically changed to Thai font
\setotherlanguage{thai}

\AtBeginDocument\captionsthai               % Force the caption to Thai

%%%%%%%%%%%%%%%%%%%%%%%%%%%%%%%%%%%%%%%%%%%%%%%%%%%%%%%%%%%%%%%%%%%%%%%%%%

\theoremstyle{definition}
\newtheorem{problem}{ปัญหาข้อที่}
\newtheorem{definition}{บทนิยาม}[chapter]
\newtheorem{example}{ตัวอย่าง}[section]
\newtheorem{theorem}{ทฤษฎีบท}[chapter]
\newtheorem{lemma}{บทตั้ง}[section]
\newtheorem{corollary}{บทแทรก}[section]
%\newtheorem{problem}{โจทย์ข้อที่}

\theoremstyle{remark}
%\newtheorem*{remark}{ข้อสังเกต}
\newtheorem*{note}{สังเกต}



\title{การพิสูจน์เบื้องต้น}

%\makeatletter
%\def\@maketitle{%
%	\newpage%
%	\vspace{15 in}%
%	\begin{center}%
%		{\color{black}\Huge\bfseries\@title \par}%
%		\vskip 1.5em%
%	\end{center}%
%	\par
%%	\vskip 1.5em
%	}
%\makeatother
%
\usepackage[explicit]{titlesec}
%\makeatletter
%\def\@makechapterhead#1{%
%	%%%%\vspace*{50\p@}% %%% removed!
%	{\parindent \z@ \raggedright \normalfont
%		\ifnum \c@secnumdepth >\m@ne
%		\huge\bfseries \@chapapp\space \thechapter
%		\par\nobreak
%		\vskip 20\p@
%		\fi
%		\interlinepenalty\@M
%		\Huge \bfseries #1\par\nobreak
%		\vskip 10\p@
%}}
%\def\@makeschapterhead#1{%
%	%%%%%\vspace*{50\p@}% %%% removed!
%	{\parindent \z@ \raggedright
%		\normalfont
%		\interlinepenalty\@M
%		\Huge \bfseries  #1\par\nobreak
%		\vskip 10\p@
%}}
%\makeatother
%
%\usepackage{tikz}\usetikzlibrary{shapes.misc}
%\newcommand{\titlebar}[1]{%
%	\tikz[baseline,trim left=1in,trim right=1in] {
%		\fill [yellow!35] (1in,-1.5ex) rectangle (#1 + 1.12in,3.5ex);
%	}%
%}
%\titleformat{name=\section,numberless}{\bfseries\large}{}{0in}%
%{\addcontentsline{toc}{section}{#1}%
%	\settowidth{\mylength}{#1}\titlebar{\mylength} #1}
%
%\newmdenv[%
%backgroundcolor = RoyalBlue,%
%hidealllines=true]{extr}
%

\titleformat
{\chapter}
[block]
{\bfseries\Huge}
{}
{0in}
{#1 \hfill \thechapter}
[\vspace{-2ex}%
\color{BurntOrange}\rule{\textwidth}{3pt}]

%\titleformat
%{\chapter} % command
%[display] % shape
%{\bfseries\Large\itshape} % format
%{Story No. \ \thechapter} % label
%{0.5ex} % sep
%{
%	\rule{\textwidth}{1pt}
%	\vspace{1ex}
%	\centering
%	#1
%} % before-code
%[
%\vspace{-0.5ex}%
%\rule{\textwidth}{0.3pt}
%] % after-code

%\BeforeBeginEnvironment{equation*}{\begin{singlespace}}
%	\AfterEndEnvironment{equation*}{\end{singlespace}\noindent\ignorespaces}
%\BeforeBeginEnvironment{align}{\begin{onehalfspace}}
%	\AfterEndEnvironment{align}{\end{onehalfspace}\noindent\ignorespaces}

\newcommand{\q}[1]{``#1''}

\newcommand{\fitch}[1]{
	
\begin{minipage}[l]{0.5in}%
		\begin{equation*}%
		\begin{nd}%
		#1	%
		\end{nd}%
		\end{equation*}%
		\vspace{0pt}%
\end{minipage}%

}

\makeatletter
\g@addto@macro\normalsize{%
	\setlength\abovedisplayskip{5pt}
	\setlength\belowdisplayskip{5pt}
	\setlength\abovedisplayshortskip{5pt}
	\setlength\belowdisplayshortskip{5pt}
}
\makeatother

\mdfdefinestyle{mystyle}{leftmargin=1.5in,linecolor=blue,innerleftmargin=0.5in,innerleftmargin=0.5in,linewidth=2pt,rightmargin=1.5in}
\newmdenv[%
leftmargin=0.5in,rightmargin=0.5in,usetwoside=false]{textbox}

\newcommand{\boxthis}[1]{
	\begin{textbox}%
		#1 
	\end{textbox}%
}

\DeclarePairedDelimiter{\Set}{\lbrace}{\rbrace}

\DeclareMathOperator{\cod}{cod}
\DeclareMathOperator{\dom}{dom}
\tikzcdset{every label/.append style = {font = \small}}
\begin{document}
		\begin{titlepage}
			\begin{flushright}
				\vspace*{-1.5in}
				\begin{minipage}[b]{0.7\linewidth}
					\hfil\begingroup\fontsize{36}{44}\selectfont\bfseries ทฤษฎีของประเภท\\Category Theory{ }\endgroup
				\end{minipage}
				\begin{minipage}[b]{2pt}
					\color{BurntOrange}\rule{6pt}{7in}
				\end{minipage}	
				\vskip 1.5em
				
				\vspace*{2in}
				ฉบับร่าง วันที่ \makeatletter\@date\makeatother
				\thispagestyle{empty}
				\pagenumbering{gobble}
			\end{flushright}
			\pagenumbering{roman}
			\tableofcontents
		\end{titlepage}
		\pagenumbering{arabic}
		
		\chapter{บทนำ}
		\begin{definition}
			แคเทกอรี (Category) ประกอบด้วย
			\begin{itemize}
				\item ลูกศร แทนด้วยสัญลักษณ์ $f, g, h, \ldots$
				\item วัตถุ แทนด้วยสัญลักษณ์ $A, B, C, \ldots$
				\item แต่ละลูกศร $f$ จะมีวัตถุที่เกี่ยวข้องอยู่ด้วย คือ $\dom(f), \cod(f)$ เรียกว่า \textit{โดเมน} และ \textit{โคโดเมน} ของ $f$ ตามลำดับ นิยมเขียนด้วยสัญลักษณ์
				\begin{equation*}
					f\colon A \rightarrow B
				\end{equation*}
				และให้เข้าใจว่า $A = \dom(f)$ และ $B = \cod(f)$
				\item สำหรับ $f, g$ ที่มีสมบัติว่า $\cod(f) = \dom(g)$ จะมีลูกศร
				\begin{equation*}
				g\circ f \colon \dom(f) \to \cod(g)
				\end{equation*}
				เรียกลูกศรนี้ว่า\textit{คอมโพสิท}ของ $f$ และ $g$
				\item สำหรับแต่ละวัตถุ $A$ จะมีลูกศรไปหาตัวมันเอง
				\begin{equation*}
					1_A \colon A \to A
				\end{equation*}
				เรียกว่า \textit{ลูกศรเอกลักษณ์} ของ $A$
			\end{itemize}
				นอกจากนี้แล้ว แคเทกอรีต้องสอดคล้องกับเงื่อนไขดังต่อไปนี้
					\begin{itemize}
						\item (สมบัติการจัดกลุ่ม) $h \circ (g \circ f) = (h \circ g) \circ f$ สำหรับลูกศรใด ๆ ที่ซึ่ง $f \colon A \to B, g\colon B \to C, h\colon C \to D$ 
						\item (สมบัติเอกลักษณ์) $f\circ 1_A = f = 1_A \circ f$ สำหรับทุก ๆ $f \colon A \to B$
					\end{itemize}
		\end{definition}
	
		\chapter{แคเทกอรีเล็กในบริเวณ}
		\section{บทตั้งของโยเนดะ}
		
		\chapter{การแต่งเติม}
		
		\chapter{ลิมิตและลิมิตร่วม}
		
		\begin{definition}
			สำหรับแคเทกอรี $\mathbf C$ ใด ๆ \textit{พุลแบ็ค} (Pullback) ของลูกศร $f, g$ ที่ซึ่ง $\cod(f) = \cod(g)$
			คือลูกศร $p_1, p_2$ ที่ทำให้ $fp_1 = gp_2$ ดังแผนภาพต่อไปนี้
			\begin{equation*}
			\begin{tikzcd}[column sep = huge, row sep=huge]
			& P \arrow[d, "p_1"] \arrow[r, "p_2"'] & B \arrow[d, "g"] \\
			& A \arrow[r, "f"']                    & C               
			\end{tikzcd}
			\end{equation*}
			 และสมบัตินี้เป็นสากล นั่นคือสำหรับ $z_1\colon Z \to A$ และ $z_2\colon Z \to B$ ที่มีสมบัติว่า $z_1f = z_2g$ เช่นกัน จะมี $u \colon Z \to P$ เพียงหนึ่งเดียว ซึ่งมีสมบัติว่า $z_1 = p_1u$ และ $z_2 = p_2u$ เป็นไปตามแผนภาพด้านล่าง
			 
			\begin{equation*}
			\begin{tikzcd}[column sep = huge, row sep=5.0em]
				Z \arrow[rd, "u" description, dotted] \arrow[rrd, "z_2"] \arrow[rdd, "z_1"'] &                                      &                  \\
				& P \arrow[d, "p_1"] \arrow[r, "p_2"'] & B \arrow[d, "g"] \\
				& A \arrow[r, "f"']                    & C               
			\end{tikzcd}
			\end{equation*}
		\end{definition}
		จะเห็นว่าพุลแบ็คมีได้เพียงแบบเดียวขึ้นกับไอโซมอร์ฟิสซึม
		
		\section{ทฤษฎีบทฟังก์เตอร์แต่งเติม}
		
		\chapter{โมแนด}
\end{document}