% !TeX spellcheck = th_TH
\documentclass[a4paper,12pt]{extbook}

%%%%%%%%%%%%%%%%%%%%%%
\usepackage[dvipsnames]{xcolor}
\usepackage[framemethod=TikZ]{mdframed}

\usepackage{fancyhdr}
\usepackage{multicol, array, tabularx, enumitem}
%%%%%%%%%%%%%%%%%%%%%%



% Set the locale for linebreak to Thai
\XeTeXlinebreaklocale "th"
% In English, when TeX tries to justify text,
% it will add some spaces between words
% For Thai, we "must not" add any space between words
% i.e. put "zero" space between words
\XeTeXlinebreakskip = 0pt plus 0pt
% For a bit better(?) justified output
\sloppy

\usepackage[top = 1in, bottom = 1in, left=1in, right = 1in]{geometry}

\usepackage{amsthm, amsmath, amssymb, mathtools, indentfirst}


\usepackage[sb]{libertinus}

% For any unicode characters, require XeTeX/XeLaTeX
\usepackage{fontspec}
\defaultfontfeatures{Mapping=tex-text} 

% Set main fonts
% For Thai, I recommend to scale the size to the uppercase size of latin alphabet
%\setmainfont[Scale=MatchLowercase,Mapping=tex-text]{TH Sarabun New}
%%\setmainfont{TeX Gyre Termes}				% Free Times
%%
%%% Sans font
%%\setsansfont{TeX Gyre Heros}				% Free Helvetica
%%
%%% Monospace font
%%\setmonofont{TeX Gyre Cursor}				% Free Courier
%\SetSymbolFont{operators}{normal}{OT1}{tex-gyre-pagella}{m}{n}
%\DeclareSymbolFont{symbols}{OMS}{cmsy}{m}{n}
%\DeclareSymbolFont{largesymbols}{OMX}{cmex}{m}{n}

\setmathfont{LibertinusMath-Regular.otf}

%% Because latin font in Sarabun is Sans Serif, we prefer to use Serif font
\newfontfamily{\thaifont}[Scale=MatchLowercase,Mapping=tex-text]{TH Sarabun New}

% Set environment for Thai fonts
\newenvironment{thailang}
{\thaifont}
{}

% For automatic switching between languages
\usepackage[Latin,Thai]{ucharclasses}

% When using Thai characters use thailang environment
\setTransitionTo{Thai}{\begin{thailang}}
% For other characters, switch back to the original environment
\setTransitionFrom{Thai}{\end{thailang}}

% Single spacing is too tight for Thai
\usepackage[nodisplayskipstretch]{setspace}
\onehalfspacing

\usepackage{etoolbox}

% For thaialph numbering \thAlph
\usepackage{polyglossia}          
% Set the normal language to English
% i.e. numbering, latin characters will use English font
\setdefaultlanguage{english}
% When using Thai characters, the font will be automatically changed to Thai font
\setotherlanguage{thai}

\AtBeginDocument\captionsthai               % Force the caption to Thai

%%%%%%%%%%%%%%%%%%%%%%%%%%%%%%%%%%%%%%%%%%%%%%%%%%%%%%%%%%%%%%%%%%%%%%%%%%

\theoremstyle{definition}
\newtheorem{problem}{ปัญหาข้อที่}
\newtheorem{definition}{บทนิยาม}[section]
\newtheorem{example}{ตัวอย่าง}[section]
\newtheorem{theorem}{ทฤษฎีบท}[chapter]
\newtheorem{lemma}{บทตั้ง}[section]
\newtheorem{corollary}{บทแทรก}[section]
%\newtheorem{problem}{โจทย์ข้อที่}

\theoremstyle{remark}
%\newtheorem*{remark}{ข้อสังเกต}
\newtheorem*{note}{สังเกต}



\title{การพิสูจน์เบื้องต้น}

%\makeatletter
%\def\@maketitle{%
%	\newpage%
%	\vspace{15 in}%
%	\begin{center}%
%		{\color{black}\Huge\bfseries\@title \par}%
%		\vskip 1.5em%
%	\end{center}%
%	\par
%%	\vskip 1.5em
%	}
%\makeatother
%
\usepackage[explicit]{titlesec}
%\makeatletter
%\def\@makechapterhead#1{%
%	%%%%\vspace*{50\p@}% %%% removed!
%	{\parindent \z@ \raggedright \normalfont
%		\ifnum \c@secnumdepth >\m@ne
%		\huge\bfseries \@chapapp\space \thechapter
%		\par\nobreak
%		\vskip 20\p@
%		\fi
%		\interlinepenalty\@M
%		\Huge \bfseries #1\par\nobreak
%		\vskip 10\p@
%}}
%\def\@makeschapterhead#1{%
%	%%%%%\vspace*{50\p@}% %%% removed!
%	{\parindent \z@ \raggedright
%		\normalfont
%		\interlinepenalty\@M
%		\Huge \bfseries  #1\par\nobreak
%		\vskip 10\p@
%}}
%\makeatother
%
%\usepackage{tikz}\usetikzlibrary{shapes.misc}
%\newcommand{\titlebar}[1]{%
%	\tikz[baseline,trim left=1in,trim right=1in] {
%		\fill [yellow!35] (1in,-1.5ex) rectangle (#1 + 1.12in,3.5ex);
%	}%
%}
%\titleformat{name=\section,numberless}{\bfseries\large}{}{0in}%
%{\addcontentsline{toc}{section}{#1}%
%	\settowidth{\mylength}{#1}\titlebar{\mylength} #1}
%
%\newmdenv[%
%backgroundcolor = RoyalBlue,%
%hidealllines=true]{extr}
%

\titleformat
{\chapter}
[block]
{\bfseries\Huge}
{}
{0in}
{#1 \hfill \thechapter}
[\vspace{-2ex}%
\color{RoyalBlue}\rule{\textwidth}{2pt}]

%\titleformat
%{\chapter} % command
%[display] % shape
%{\bfseries\Large\itshape} % format
%{Story No. \ \thechapter} % label
%{0.5ex} % sep
%{
%	\rule{\textwidth}{1pt}
%	\vspace{1ex}
%	\centering
%	#1
%} % before-code
%[
%\vspace{-0.5ex}%
%\rule{\textwidth}{0.3pt}
%] % after-code

%\BeforeBeginEnvironment{equation*}{\begin{singlespace}}
%	\AfterEndEnvironment{equation*}{\end{singlespace}\noindent\ignorespaces}
%\BeforeBeginEnvironment{align}{\begin{onehalfspace}}
%	\AfterEndEnvironment{align}{\end{onehalfspace}\noindent\ignorespaces}

\newcommand{\q}[1]{``#1''}

\newcommand{\fitch}[1]{
	
\begin{minipage}[l]{0.5in}%
		\begin{equation*}%
		\begin{nd}%
		#1	%
		\end{nd}%
		\end{equation*}%
		\vspace{0pt}%
\end{minipage}%

}

\makeatletter
\g@addto@macro\normalsize{%
	\setlength\abovedisplayskip{5pt}
	\setlength\belowdisplayskip{5pt}
	\setlength\abovedisplayshortskip{5pt}
	\setlength\belowdisplayshortskip{5pt}
}
\makeatother

\usepackage{fitch}

\mdfdefinestyle{mystyle}{leftmargin=1.5in,linecolor=blue,innerleftmargin=0.5in,innerleftmargin=0.5in,linewidth=2pt,rightmargin=1.5in}
\newmdenv[%
leftmargin=0.5in,rightmargin=0.5in,usetwoside=false]{textbox}

\newcommand{\boxthis}[1]{
	\begin{textbox}%
		#1 
	\end{textbox}%
}

\DeclarePairedDelimiter{\Set}{\lbrace}{\rbrace}



\begin{document}

\begin{titlepage}
	\begin{flushright}
		\vspace*{-1.5in}
		\begin{minipage}[b]{0.53\linewidth}
			\begingroup\fontsize{36}{44}\selectfont\bfseries การพิสูจน์เบื้องต้น{ }\endgroup
		\end{minipage}
		\begin{minipage}[b]{2pt}
			\color{RoyalBlue}\rule{6pt}{7in}
		\end{minipage}	
		\vskip 1.5em
		{\fontsize{20}{44} และระบบการพิสูจน์ธรรมชาติ{  }}
		
		\vspace*{2in}
		ฉบับร่าง วันที่ \makeatletter\@date\makeatother
		\thispagestyle{empty}
		\pagenumbering{gobble}
	\end{flushright}
	\pagenumbering{roman}
	\tableofcontents
\end{titlepage}
	\pagenumbering{arabic}
	\part{ตรรกศาสตร์}
	\chapter{แนวคิดพื้นฐานในตรรกวิทยา}
		\section{การอ้างเหตุผล}
		ตรรกวิทยาเป็นวิชาที่ศึกษาการอ้างเหตุผล จำแนกการอ้างเหตุผลที่ดีและไม่ดี และให้หลักการแก่การอ้างเหตุผลเพื่อรองรับให้การอ้างเหตุผลนั้นประสบความสำเร็จ แต่การอ้างเหตุผลคืออะไร
		
		ก่อนจะให้คำนิยามที่รัดกุม ขอให้ผู้อ่านพิจารณาข้อความด้านล่าง แล้วเปรียบเทียบข้อความทั้งสอง
		\begin{itemize}
			\item สภาพเศรษฐกิจในระยะนี้อยู่ในช่วงชะลอตัว ประกอบกับค่าเงินบาทที่แข็งขึ้นเมื่อเทียบกับเงินสกุลอื่น สภาพข้างต้นทำให้การส่งออกถดถอยลง
			
			\item ธนาคารแห่งประเทศไทยมักจะมีนโยบายผ่อนคลายทางการเงินเมื่อเศรษฐกิจชะลอตัว เราทราบกันว่าค่าเงินบาทแข็งขึ้นทำให้เศรษฐกิจชะลอตัว และเหตุการณ์ที่เราประสบในตอนนี้คือค่าเงินบาทแข็งขึ้นมาก ดังนั้นธนาคารแห่งประเทศไทยต้องมีนโยบายผ่อนคลายทางการเงินแน่นอน
		\end{itemize}
		ผู้อ่านจะเห็นว่า ในข้อความแรกเป็นเพียงข้อความบรรยายสภาพของเหตุการณ์ หรือธรรมชาติต่าง ๆ ที่เกิดขึ้น ไม่มีการใช้เหตุผลเพื่อเชื่อมโยงระหว่างข้อความ ในขณะที่ข้อความที่สองมีการใช้คำเพื่อแสดงให้เห็นว่าเหตุการณ์หนึ่งจะต้องเกิดขึ้นตามมาโดยเป็นผลจากเหตุการณ์อื่น (``ธนาคารแห่งประเทศไทยต้องมีนโยบายผ่อนคลายทางการเงิน") การใช้ข้อความและโน้มน้าวให้เห็นความเชื่อมโยงจากข้อความแวดล้อมอื่น ๆ คือ \textit{การอ้างเหตุผล} (argument)
		
		ข้อความที่สอง คำว่า \q{ดังนั้น} บ่งบอกว่าประโยคที่ตามมาเป็น\textit{ข้อสรุปของการอ้างเหตุผล} (conclusion) ข้อความก่อนหน้าเป็น\textit{ข้ออ้างของการอ้างเหตุผล} (premise) ถ้าเราเชื่อข้ออ้างของการอ้างเหตุผล ดูเหมือนว่าเราจำเป็นต้องเชื่อในข้อสรุปของการอ้างเหตุผลข้างต้น โดยอาศัยหลักเหตุผลที่เราพอจะทราบคร่าว ๆ ได้
		
		ข้อสังเกตประการหนึ่งคือ ข้อสรุปไม่จำเป็นต้องอยู่ท้ายประโยคก็ได้
		\begin{center}
			คุณควรพกร่มไปด้วยนะ ตอนนี้ฝนกำลังตกหนัก และถ้าคุณไม่พกร่มไปด้วย คุณก็จะเปียกนะสิ
		\end{center}
		แต่อย่างไรเสียเราก็สามารถเรียงประโยคให้ข้อสรุปอยู่ท้ายสุด
		\begin{center}
			 ตอนนี้ฝนกำลังตกหนัก และถ้าคุณไม่พกร่มไปด้วย คุณก็จะเปียก เพราะฉะนั้นคุณควรพกร่มไปด้วย
		\end{center}
		ฉะนั้นเราอาจกล่าวได้ว่าการอ้างเหตุผลคือการเรียบเรียงประโยคโดยมีข้อสรุป และประโยคอื่น ๆ เป็นข้ออ้าง นิยามดังกล่าวไม่สนใจว่าข้อสรุปจะเป็นเหตุผลจากข้ออ้างหรือไม่ ฉะนั้นประโยคด้านล่างยังคงเป็นการอ้างเหตุผลเช่นกัน
		\begin{itemize}
			\item [] มีนักดนตรีเพียงคนเดียวในวงเครื่องสาย
			\item [] ดังนั้นชาตรีเป็นคนร่าเริง
		\end{itemize}
		ตัวอย่างข้างต้นถือว่าเป็นการอ้างเหตุผลเช่นกัน แต่เป็นการอ้างเหตุผลที่ไม่ถูกต้องอย่างมาก ทำไม?
		
		ผู้อ่านอาจสังเกตว่า
		\begin{enumerate}
			\item \textbf{ข้ออ้างไม่เป็นจริง} วงเครื่องสายมีนักดนตรีหลายคน มิฉะนั้นก็ไม่เป็นวงเครื่องสาย
			\item \textbf{ข้อสรุปไม่ได้เป็นผลจากข้ออ้าง} ชาตรีเป็นคนร่าเริง \textbf{ไม่ได้}เกี่ยวข้องกับจำนวนนักดนตรีในวงเครื่องสาย
		\end{enumerate}
		ปัญหาว่าข้ออ้างเป็นจริงหรือไม่ ไม่ใช่ปัญหาของนักตรรกวิทยา! คนที่เหมาะสมที่จะสรุปได้ว่าข้ออ้างเป็นจริงหรือไม่ คือนักดนตรี (หรือนักวิทยาศาสตร์, นักประวัติศาสตร์ ฯลฯ) สิ่งที่เราสนใจคือปัญหาข้อที่สอง: \textit{ข้อสรุปเป็นผลจากข้ออ้างหรือไม่} นี่คือสิ่งที่เราจะสนใจในวิชานี้
		
		\section{ความสมเหตุสมผล}
			ฉะนั้นแล้วเราจึงสนใจว่าเมื่อไหร่ที่ข้อสรุปของการอ้างเหตุผลจะเป็นผลมาจากข้ออ้างของการอ้างเหตุผล ในอีกแง่หนึ่ง สิ่งที่เราสนใจคือ ถ้าข้ออ้างทั้งหมดเป็นจริง แล้วข้อสรุปจะต้องเป็นจริงด้วยหรือไม่ จึงเป็นที่มาของนิยามด้านล่าง
			\boxthis{การอ้างเหตุผลจะ\textit{สมเหตุสมผล} ก็ต่อเมื่อ เป็นไปไม่ได้ที่ข้ออ้างเป็นจริง และข้อสรุปเป็นเท็จ}
			ลองพิจารณาตัวอย่างด้านล่าง
			\begin{itemize}
				\item[] คนมีปีก หรือไม่เช่นนั้นคนเหาะเหินเดินอากาศได้
				\item[]	คนไม่มีปีก
				\item[] ดังนั้น คนเหาะเหินเดินอากาศได้
			\end{itemize}
			ข้อสรุปของการอ้างเหตุผลนี้เป็นเรื่องไร้สาระ! \textit{แต่เป็นผลมาจากข้ออ้าง} ถ้าข้ออ้างเป็นจริง นั่นคือคนมีปีก หรือไม่เช่นนั้นคนเหาะเหินเดินอากาศได้ เป็นจริง และเนื่องจากข้อความที่ว่า ``คนไมมีปีก'' เป็นจริง จึงบังคับให้ส่วนที่เหลือของข้ออ้างข้อแรกที่ว่า คนเหาะเหินเดินอากาศได้ เป็นจริง ดังนั้นข้อสรุปนี้เป็นจริง ดังนั้นการอ้างเหตุผลนี้สมเหตุสมผล
			
			อีกตัวอย่างหนึ่ง
			\begin{itemize}
				\item[] คนทุกคนต้องตาย
				\item[]	โสเกรตีสเป็นคน
				\item[] ดังนั้น กรุงเทพเป็นเมืองหลวงของประเทศไทย
			\end{itemize}
			ข้ออ้างและข้อสรุปข้างต้นเป็นจริงทั้งหมด แต่การอ้างเหตุผลนี้กลับไม่สมเหตุสมผล เพราะข้อสรุปไม่จำเป็นต้องเป็นจริง กรุงธนบุรีอาจจะเป็นเมืองหลวงของประเทศไทยก็ได้ ถึงแม้ข้ออ้างทั้งสองจะยังจริงอยู่ทั้งคู่
			
			จากตัวอย่างข้อต้น ข้อสำคัญคือความสมเหตุสมผลของการอ้างเหตุผล ขึ้นอยู่กับว่าข้อสรุป\textbf{ต้อง}เป็นจริง ถ้าข้ออ้างทั้งหมดเป็นจริง และไม่เกี่ยวกับความจริงของข้ออ้างหรือข้อสรุปเลย แต่อย่างไรเสียเรามีชื่อเรียกให้กับการอ้างเหตุผลที่สมเหตุสมผล และข้ออ้างทั้งหมดเป็นจริงว่า การอ้างเหตุผลนั้นเป็นการอ้างเหตุผลที่\textit{ถูกต้อง} (sound)
			
			\section{การอุปนัยและการนิรนัย}
			พิจารณาการอ้างเหตุผลต่อไปนี้
			\begin{itemize}
				\item [] เดือนพฤษภาคม ปี 2550 ฝนตกที่นี่
				\item [] เดือนพฤษภาคม ปี 2551 ฝนตกที่นี่
				\item [] เดือนพฤษภาคม ปี 2552 ฝนตกที่นี่
				\item [] เดือนพฤษภาคม ปี 2553 ฝนตกที่นี่
				\item [] เดือนพฤษภาคม ปี 2554 ฝนตกที่นี่
				\item [] ดังนั้นฝนตกที่นี่ทุก ๆ เดือนพฤษภาคมในแต่ละปี
			\end{itemize}
			การอ้างเหตุผลข้างต้นเป็นการขยายข้อสรุปจากข้ออ้าง เรียกว่า \textit{การอุปนัย}\footnote{การอุปนัยในที่นี้แตกต่างจากการอุปนัยเชิงคณิตศาสตร์ (mathematical induction) อันที่จริงแล้วการอุปนัยเชิงคณิตศาสตร์เป็นการอ้างเหตุผลแบบนิรนัย เพียงแต่มีลักษณะ\textit{คล้าย}กับการอุปนัยจึงได้ชื่อว่าเป็นการอุปนัย} (induction) เป็นการขยายกรณีที่เกิดขึ้น หรือข้อเท็จจริงในปัจจุบันให้เป็นทั่วไปยิ่งขึ้น ในขณะที่ตัวอย่างก่อน ๆ เป็นการอ้างเหตุผลแบบนิรนัย (deduction) สังเกตว่า การอุปนัยไม่มีหลักฐานหรือเหตุผลรองรับว่าจะต้องเป็นจริง แม้ว่าเราจะมีตัวอย่างเพิ่มจากนี้มากเท่าไรก็ตาม เพราะยังคงเป็นไปได้ที่ฝนจะ\textit{ไม่ตก}ในเดือนพฤษภาคม การให้เหตุผลกับการอุปนัยจึงยุ่งยากและซับซ้อนกว่ากันมาก ในหนังสือเล่มนี้จึงพิจารณาเฉพาะการนิรนัยเท่านั้น
			\section{ค่าความจริง}
			ประโยคจำนวนมากในภาษาธรรมชาติของมนุษย์ไม่สามารถใช้เป็นข้ออ้างได้ เช่น
			\begin{itemize}
				\item \textbf{คำถาม} เช่น \q{ตอนนี้หิวไหม}
				\item \textbf{คำสั่ง} 	เช่น \q{นั่งลง}
				\item \textbf{คำอุทาน}	เ่ช่น \q{โอ๊ย!}
			\end{itemize}
			ทั้งนี้เป็นเพราะว่า ประโยคเหล่านี้ไม่ได้กล่าวเสนอความจริงของสิ่งใด ดังนั้นจึงไม่สามารถเป็นจริงหรือเท็จได้ 
			
			ในกรณีของประโยคคำถาม สิ่งที่เป็นจริงหรือเท็จคือ \textit{คำตอบ} ของคำถาม ไม่ใช่\textit{ตัวคำถาม}โดยตรง (\q{ตอนนี้หิวไหม} \q{ผิดแล้ว!}) 
			
			ต่อไปนี้เมื่อเรากล่าวถึงข้ออ้างและข้อสรุปของการอ้างเหตุผล ประโยคเหล่านั้นต้องเป็นจริงอย่างใดอย่างหนึ่งเท่านั้น อีกนัยหนึ่งคือ ประโยคนั้นมีค่าความจริง (truth value) ที่แน่นอน
			
			\section{ความต้องกัน}
			ลองพิจารณาประโยคต่อไปนี้
			\begin{itemize}
				\item[(A1)]  สมศรีมีอายุมากกว่าสมศักดิ์
				\item[(A2)]  สมศักดิ์มีอายุมากกว่าสมศรี
			\end{itemize}
			เราไม่มีทางรู้ว่าประโยคใดในสองประโยคนี้เป็นจริงได้โดยอาศัยเพียงหลักตรรกศาสตร์ แต่เราทราบว่าถ้าประโยคหนึ่งเป็นจริง อีกประโยคหนึ่งต้องเป็นเท็จ เช่นถ้า (A1) จริง แล้ว (A2) ต้องเป็นเท็จ ในทางกลับกัน ถ้า (A2) จริง แล้ว (A1) ก็ต้องเป็นเท็จด้วย ประโยคทั้งสองขัดแย้งกันและไม่มีทางเป็นจริงได้พร้อมกันทั้งคู่ ดังนั้นเราจึงให้นิยามต่อไปนี้
			\boxthis{เราจะเรียกประโยคชุดหนึ่งว่า \textit{ต้องกัน} หรือ\textit{มีความต้องกัน} (consistency) เมื่อประโยคทั้งหมดเป็นจริงพร้อมกันได้}
			และในทางกลับกัน จะเรียกว่า (A1)  และ (A1) ว่าเป็นประโยคไม่ต้องกัน หรือประโยคขัดแย้งกัน
			
			\section{ความจำเป็นและความไม่จำเป็น}
				ประโยคบางประโยคอาจเป็นจริงเสมอก็ได้ พิจารณาตัวอย่างด้านล่าง
				\begin{enumerate}
					\item ตอนนี้แดดออก
					\item ตอนนี้แดดออก หรือไม่ก็ตอนนี้แดดไม่ออก
					\item ตอนนี้ทั้งแดดออกและแดดไม่ออก
				\end{enumerate}
				ประโยคแรกตรวจสอบได้โดยการมองออกไปนอกหน้าต่าง ผู้อ่านเห็นได้ชัดว่า ประโยคแรกไม่จำเป็นต้องเป็นจริง เพราะขณะที่พูดนั้น ฝนอาจจะตก หรือเป็นเวลากลางคืนก็ได้ ค่าความจริงของมันเปลี่ยนไปตามสภาพของสิ่งอื่น 
				
				ประโยคที่ 2 นั้นแตกต่างออกไปจากประโยคแรก ไม่ว่าสภาพอากาศจะเป็นอย่างไร ประโยคที่สองย่อมเป็นจริงเสมอ เราเรียกว่าประโยคที่จริงเสมอว่า ประโยคจริงโดยจำเป็น (necessary truth)
				
				ในขณะที่ประโยคที่สามต้องเป็นเท็จแน่นอนโดยไม่ขึ้นกับสภาพอากาศเลย แดดอาจจะออกตอนที่เริ่มต้นพูดประโยคนี้ และแดดไม่ออกเมื่อพูดประโยคนี้เสร็จ แต่ไม่มีทางที่แดดออกและแดดไม่ออกในเวลาเดียวกันและในสถานที่เดียวกัน เราเรียกประโยคที่เป็นเท็จเสมอว่า ประโยคเท็จโดยจำเป็น (necessary falsehood)
				
				ประโยคที่ไม่จำเป็นต้องเป็นจริงหรือเท็จ เรียกว่า ประโยคไม่จำเป็น (contingent)
			\section{แบบฝึกหัด}
			\begin{enumerate}
				\item ข้อความต่อไปนี้เป็นการอ้างเหตุผลหรือไม่ ถ้าใช่แล้วจงหาข้อสรุปของการอ้างเหตุผล
				\begin{enumerate}[label={(\arabic*)}]
					\item ขณะนี้แดดออก ฉะนั้นฉันควรทาครีมกันแดด
					\item เมื่อวานแดดต้องออกแน่ ๆ เลย เพราะฉันพกร่มไปด้วย
					\item มีผู้ให้คำนิยามว่า ทุนนิยมเป็นระบบเศรษฐกิจที่เอกชนเป็นเจ้าของการค้า การผลิตและอุตสาหกรรมทั้งหมด หน้าที่ของรัฐเป็นไปเพื่อดูแลความสงบในส่วนอื่น ๆ ของสังคม และรัฐมีส่วนน้อยมากในวงการผลิต อย่างไรเสียนักเศรษฐศาสตร์หลายคนต่างยอมรับว่า หน้าที่ของรัฐอาจจะต้องมีมากกว่าหน้าที่ดูแลให้เกิดตลาดเสรีในระบบ บางทีรัฐอาจจะต้องเข้ามาเป็นผู้เล่นหนึ่ง หรือเป็นผู้ดูแลกำหนดกฎเกณฑ์ให้กับตลาด เพื่อให้เกิดความยุติธรรมในโลกที่ไม่มีอะไรสมบูรณ์แบบ
					\item ปัจจัยหนึ่งที่ทำให้ยุโรปเกิดการตื่นรู้ในศตวรรษที่ 18 คือความก้าวหน้าของการตีพิมพ์ในยุโรป ปฏิเสธไม่ได้ว่า การที่หนังสือสามารถมีได้อย่างมาก และราคาถูกลงเป็นผลดีต่อการพัฒนาความคิดในด้านต่าง ๆ ของชาวยุโรป จำนวนหนังสือที่มากขึ้น มาพร้อมกับอัตราการรู้หนังสือที่เพิ่มขึ้นอย่างมาก และการติดต่อสื่อสารที่สะดวกรวดเร็วขึ้นผ่านระบบไปรษณีย์ทำให้ยุโรปเกิดกระแสตื่นรู้ขึ้นมาในเวลาไล่เลี่ยกันจนถึงปลายศตวรรษที่ 18 นั้นเอง
					\item จะเห็นได้ว่า $\triangle ABC$ เป็นสามเหลี่ยมหน้าจั่วที่ $A$ เป็นจุดยอดร่วมของด้านที่เท่ากัน ดังนั้น $AB = AC$ ซึ่งส่งผลให้ $AB^2 = AC^2$
					\item สายลมเชยรำเพยพา กลิ่นเอามาให้เราดม เดี๋ยวนี้ซิหนอ ยังสู้แตกกอไว้ให้ชื่นชม สุดเสียดายเขาเด็ดดอกเอาไปดม อกเราต้องระทม เพียงต้องสายลมยังเรรวน
					\item ดูสิ! เศษคุกกี้กระจายเต็มห้องครัวไปหมด ตอนนี้แกก็อยู่ในห้องครัวคนเดียว แถมยังถือโถคุกกี้ไว้อีก แกต้องเป็นคนแอบกินขนมแน่ ๆ
				\end{enumerate}
				\item การอ้างเหตุผลต่อไปนี้สมเหตุสมผลหรือไม่
				\begin{enumerate}[label={(\arabic*)}]
					\item \begin{enumerate}[label={\arabic*.}]
						\item คนทุกคนเป็นบร็อคโคลี่
						\item บร็อคโคลี่ทุกอันเป็นแครอท
						\item[] ดังนั้น คนทุกคนเป็นแครอท
					\end{enumerate}
					\item 
					\begin{enumerate}[label={\arabic*.}]
						\item ถ้าตอนนี้แดดออก ดอกบัวจะบาน
						\item ตอนนี้ฝนตก
						\item[] ดังนั้น ดอกบัวจะไม่บาน
					\end{enumerate}
					\item 
					\begin{enumerate}[label={\arabic*.}]
						\item พระเจ้ามีอยู่จริงหรือฝนกำลังตก
						\item ตอนนี้ฝนไม่ตก
						\item[] ดังนั้น พระเจ้ามีอยู่จริง
					\end{enumerate}
					\item 
					\begin{enumerate}[label={\arabic*.}]
						\item ถ้าตอนนี้แดดออก ดอกกุหลาบจะบาน
						\item ตอนนี้แดดไม่ออก
						\item[] ดังนั้น ดอกกุหลาบจะไม่บาน
					\end{enumerate}
					\item 
					\begin{enumerate}[label={\arabic*.}]
						\item ถ้าตอนนี้เป็นตอนสาย ๆ ดอกคุณนายตื่นสายจะบาน
						\item ดอกคุณนายตื่นสายไม่บาน
						\item[] ดังนั้น ตอนนี้ไม่ใช่ตอนสาย ๆ
					\end{enumerate}
					\item 
					\begin{enumerate}[label={\arabic*.}]
						\item นักคณิตศาสตร์ทุกคนชอบเล่นกีฬา
						\item มีคนเล่นกีฬาบางคนแข็งแรง
						\item[] ดังนั้น นักคณิตศาสตร์บางคนแข็งแรง
					\end{enumerate}
					\item 
					\begin{enumerate}[label={\arabic*.}]
						\item วันนี้เป็นวันที่ 1 มกราคม
						\item วันนี้เป็นวันที่ 4 มกราคม
						\item[] ดังนั้น วันนี้เป็นวันเกิดของสมศักดิ์
					\end{enumerate}
				\end{enumerate}
				\item ข้อความต่อไปนี้เป็นจริงได้หรือไม่ ถ้าได้จงยกตัวอย่าง ถ้าไม่ได้จงให้เหตุผล
				\begin{enumerate}[label={\arabic*.}]
					\item มีการอ้างเหตุผลที่สมเหตุสมผลที่มีข้ออ้างจริงข้อหนึ่ง และอีกข้อหนึ่งเป็นเท็จ
					\item มีการอ้างเหตุผลที่สมเหตุสมผลแต่ข้ออ้างเป็นเท็จทั้งหมด
					\item มีการอ้างเหตุผลที่ถูกต้อง แต่ข้อสรุปเป็นเท็จ
					\item มีการอ้างเหตุผลที่ไม่ถูกต้อง แต่สามารถเพิ่มข้ออ้างเพื่อให้สมเหตุสมผลได้
					\item มีการอ้างเหตุผลที่ถูกต้อง แต่สามารถเพิ่มข้ออ้างเพื่อให้ไม่สมเหตุสมผลได้
				\end{enumerate}
				\item จงตรวจสอบว่าประโยคต่อไปนี้ จริงโดยจำเป็น ไม่จริงโดยจำเป็น หรือไม่จำเป็น
				\begin{enumerate}[label={\arabic*.}]
					\item ออยเลอร์เป็นนักคณิตศาสตร์
					\item มีคนเป็นนักคณิตศาสตร์
					\item ไม่เคยมีใครเป็นนักคณิตศาสตร์
					\item ถ้าออยเลอร์เป็นนักคณิตศาสตร์ แล้วมีคนเป็นนักคณิตศาสตร์
					\item ถึงแม้ว่าออยเลอร์เป็นนักคณิตศาสตร์ แต่ไม่เคยมีใครเป็นนักคณิตศาสตร์
					\item ถ้าใครจะเป็นนักคณิตศาสตร์ คนนั้นก็คือออยเลอร์
					\item ไม่เคยมีใครเป็นนักคณิตศาสตร์ หรือออยเลอร์เป็นนักคณิตศาสตร์
				\end{enumerate}
				\item พิจารณาประโยคต่อไปนี้
				\begin{itemize}
					\item[B1] มีช้างอย่างน้อย 2 ตัวในสวนสัตว์
					\item[B2] มีกวาง 6 ตัวพอดีในสวนสัตว์
					\item[B3] ต้องไม่มีมนุษย์ต่างดาวเกิน 1 ตัวในสวนสัตว์
					\item[B4] ช้างทุกตัวเป็นมนุษย์ต่างดาว
				\end{itemize}
				ถ้าเลือกประโยคมาทีละสามประโยค (เช่น B1, B2, B4 หรือ B2, B3, B4) รูปแบบไหนบ้างที่ประโยคทั้งหมดต้องกัน และรูปแบบไหนบ้างที่ประโยคทั้งหมดไม่ต้องกัน
				\item ข้อความต่อไปนี้เป็นจริงได้หรือไม่ ถ้าได้จงยกตัวอย่าง ถ้าไม่ได้จงให้เหตุผล
				\begin{enumerate}[label={\arabic*.}]
					\item มีการอ้างเหตุผลที่สมเหตุสมผล แต่ข้อสรุปเป็นเท็จโดยจำเป็น
					\item มีการอ้างเหตุผลที่ไม่สมเหตุสมผล แต่ข้ออ้างเป็นจริงโดยจำเป็น
					\item มีประโยคชุดหนึ่งที่ต้องกันทั้งหมด แต่มีประโยคหนึ่งเป็นเท็จโดยจำเป็น
					\item มีประโยคชุดหนึ่งที่ไม่ต้องกันทั้งหมด แต่มีประโยคหนึ่งเป็นจริงโดยจำเป็น
				\end{enumerate}
				\item คำว่า การอ้างเหตุผล เป็นคำทับศัพท์ใหม่เพื่อทับศัพท์คำว่า argument ซึ่งไม่มีในภาษาไทย ทำไมภาษาไทยจึงไม่มีคำนี้ หรือว่าความจริงมี แล้วแสดงว่าคนไทยไม่รู้จักการอ้างเหตุผลหรือการใช้เหตุผลหรือไม่
				\item ข้อสรุปจะจริง จำเป็นที่การอ้างเหตุผลจะสมเหตุสมผลหรือไม่ หรือจำเป็นที่การอ้างเหตุผลจะถูกต้องหรือไม่ ถ้าไม่ใช่ แล้วเราจะทราบได้อย่างไรว่าข้อสรุปไหนจริง ถ้าการอ้างเหตุผลไม่เกี่ยวข้องกับความจริงของข้อสรุปเลย
				\item การอ้างเหตุผลจำเป็นต้องมีคำสันธาน เช่น \q{ดังนั้น} \q{เพราะฉะนั้น} หรือไม่
			\end{enumerate}
		
	\chapter{ตรรกศาสตร์เชิงฟังก์ชันความจริง}
		\section{ความสมเหตุสมผลทั้งสองแบบ}
		\section{ตรรกศาสตร์เชิงฟังก์ชันความจริงคืออะไร}
		\section{ประพจน์เชิงเดี่ยว}
		\section{ตัวเชื่อมประพจน์และภาษาธรรมชาติ}
		\section{ประโยคในตรรกศาสตร์เชิงฟังก์ชันความจริง}
		\section{การใช้และการกล่าวถึง} %อภิภาษา
		\section{ตารางค่าความจริง}
		\section{ตัวเชื่อมเชิงฟังก์ชันความจริง}
		\section{การให้คุณค่า}
		\section{อรรถศาสตร์ของตรรกศาสตร์เชิงฟังก์ชันความจริง}
		\subsection{สัญลักษณ์ $\vDash$}
		\section{ข้อจำกัดของตรรกศาสตร์เชิงฟังก์ชันความจริง}
		
	\chapter{ตรรกศาสตร์อันดับแรก}
		\section{ความจำเป็น}
		\section{องค์ประกอบของตรรกศาสตร์อันดับแรก} ภาคแสดง ชื่อ ตัวบ่งปริมาณ โดเมน
		\section{การแปลงภาษาธรรมชาติ}
		\section{ภาคแสดงหลากชื่อ} ลำดับของตัวบ่งปริมาณ
		\section{เอกลักษณ์} ไลป์นิซ
		\section{การบรรยายจำเพาะ}
		\section{ภาษาภาคขยาย} เอกลักษณ์
		\section{การตีความ}
		\section{หลักทางอรรถศาสตร์ของตรรกศาสตร์อันดับแรก}
	\chapter{การพิสูจน์ในระบบตรรกศาสตร์เชิงฟังก์ชันความจริง}
	บทนี้เป็นการแนะนำการพิสูจน์ประพจน์ในระบบ TFL โดยใช้การนิรนัยธรรมชาติ (Natural Deduction) 
	
%เป็นอันทราบกันดีว่าระบบการพิสูจน์มีหลากหลายระบบ อาทิ ระบบนิรนัยแบบของฮิลแบร์ท (Hilbert system) หรือระบบนิรนัยซีเควนท์ (Sequent calculi) และระบบนิรนัย
	
	ระบบนิรนัยธรรมชาติที่จะใช้ในเอกสารนี้คือ ระบบนิรนัยธรรมชาติแบบของฟิตช์ (Fitch-style natural deduction) ซึ่งริเริ่มโดยเฟรเดอริก ฟิตช์ (Frederic Fitch) ซึ่งใช้เส้นเพื่อแบ่งประพจน์ต่าง ๆ ออกจากกัน ซึ่งได้เปรียบให้ผู้อ่านเห็นภาพของการอ้างเหตุผลชัดเจนยิ่งขึ้น
	
	\section{การพิสูจน์แบบแผน}
	ในระบบธรรมชาติที่เราจะใช้ต่อไปนี้ แต่ละตัวเชื่อมประพจน์จะมีกฎการเติม (Introduction rule) เพื่อใช้ในการพิสูจน์ข้อความที่มีตัวเชื่อมประพจน์นั้น ๆ เป็นตัวเพิ่มประพจน์หลัก ในขณะเดียวกันก็มีกฎการตัด (Elimination rule) เพื่อใช้การพิสูจน์ข้อความอื่นต่อไป
	
	ก่อนอื่นจะต้องแนะนำศัพท์เฉพาะทางตรรกศาสตร์เสียก่อน นั่นคือคำว่า\textit{การพิสูจน์แบบแผน (Formal proof)} ซึ่งเป็นลำดับของประพจน์เริ่มต้นจากข้อสมมติหรือข้อตั้ง และลงท้ายด้วยข้อสรุปจากการพิสูจน์นั้น การพิสูจน์แบบแผนคือการเขียนลำดับประพจน์ให้เป็นเหตุเป็นผลกันจากข้อตั้งไปยังข้อสรุปนั่นเอง (ต่อจากนี้ขอใช้คำว่าบทพิสูจน์หรือการพิสูจน์ แทนคำว่าการพิสูจน์แบบแผนทั้งหมด)
	
	เพื่อให้เห็นภาพ จะขอยกตัวอย่างการพิสูจน์ข้อความ
	\begin{equation*}
		\neg A \vee B \,\therefore\, A \to B
	\end{equation*}
	
	ระบบแบบของฟิตช์เริ่มต้นด้วยการเขียนข้อตั้งทั้งหมด

	\begin{minipage}[l]{0.5in}
		\begin{equation*}
			\begin{nd}
				\hypo{1}{\neg A \vee B}	
			\end{nd}
		\end{equation*}%
		\vspace{0pt}
	\end{minipage}

	ทุกบรรทัดในระบบแบบของฟิตช์จะมีเลขกำกับเสมอเพื่อใช้อ้างอิง และเมื่อเขียนข้อตั้งเสร็จแล้วจะขีดเส้นกั้นใต้ข้อตั้งข้อสุดท้าย เพื่อแบ่งข้อตั้งออกจากกับประพจน์อื่น ๆ ที่จะเป็นผลตามมาจากข้อตั้ง หรือเป็นข้อตั้งเพิ่มเติมในการพิสูจน์
	
	เป้าหมายของระบบการพิสูจน์คือแสดงให้ได้ว่าผลเป็นจริง นั่นคือ 

	\begin{minipage}[l]{0.5in}
		\begin{equation*}
			\begin{nd}
				\have[n]{1}{ A \to B}	
			\end{nd}
		\end{equation*}%
		\vspace{0pt}
	\end{minipage}

	บรรทัดนี้จะต้องปรากฎในตอนท้ายของการพิสูจน์ สำหรับบางจำนวนเต็ม $n$ 
	
	ในทำนองเดียวกัน ถ้าเราต้องการพิสูจน์ข้อความที่ว่า
	\begin{equation*}
		\neg P \vee Q,\neg P \to R,\neg R \wedge S\,\therefore\, Q \wedge S
	\end{equation*}

	
	ระบบแบบของฟิตซ์ของเราจะต้องได้ว่า\nopagebreak
	\fitch{
		\hypo{1}{\neg P \vee Q}
		\hypo{2}{\neg P \to R}
		\hypo{3}{\neg R \wedge S}
		}

	และบรรทัดสุดท้ายจะต้องแสดงให้ได้ว่า
	\fitch{\have[n]{1}{	Q \wedge S}}
	ต่อไปจะเป็นการแนะนำกฎทั้งหมดของระบบแบบของฟิตช์
	\section{ตัวเชื่อมประพจน์ ``และ''}
		ในการพิสูจน์ข้อความที่เชื่อมด้วยตัวเชื่อประพจน์ ``และ'' เช่นข้อความที่ว่า
		\begin{center}
				``กรุงเทพเป็นเมืองหลวงของประเทศไทย และประเทศฝรั่งเศสอยู่ในทวีปยุโรป''
		\end{center}
		เราอาจเริ่มด้วยการพิสูจน์เสียก่อนว่ากรุงเทพเป็นเมืองหลวงของประเทศไทย และพิสูจน์ว่าประเทศฝรั่งเศสอยู่ในทวีปยุโรป จากนั้นเราสามารถนำข้อความที่ได้มาเชื่อมกันด้วยตัวเชื่อมประพจน์ ``และ'' ระบบนิรนัยธรรมชาติของเราก็สะท้อนความคิดนี้ได้เช่นกัน สมมติว่ากำหนดสัญลักษณ์แทนประพจน์ดังต่อไปนี้
		\begin{itemize}
			\item[] $B$: กรุงเทพเป็นเมืองหลวงของประเทศไทย
			\item[] $F$: ประเทศฝรั่งเศสอยู่ในทวีปยุโรป
		\end{itemize}		
		ถ้าในการพิสูจน์ของเราปรากฎว่าได้ `$B$' ในบรรทัดที่ 5 และได้ `$F$' ในบรรทัดที่ 8 เราสามารถแสดงได้ว่า `$B \wedge F$' นั่นคือ
		\fitch{
			\have[5]{1}		{B}%
			\have[8]{a}		{F}%
			\have[~]{3}		{B \wedge F}	\ai{1,a}
		}
		การเขียน $\wedge\text{I } 5, 8$ เป็นการบ่งบอกว่าบรรทัดนี้ได้จากการใช้กฎการเติมเครื่องหมาย ``และ'' (Conjuction introduction) กับบรรทัดที่ 5 และ 8 เช่นเดียวกันเราก็จะได้ว่า
		\fitch{
			\have[5]{1}		{B}%
			\have[8]{a}		{F}%
			\have[~]{3}		{F \wedge B}	\ai{a,1}
		}
		สังเกตว่าลำดับของเลขอ้างอิงเปลี่ยนไป ทั้งนี้เพื่อให้เป็นไปลำดับของประพจน์ในบรรทัดสุดท้าย
		
		{\nopagebreak เราสามารถขยายกฎการเติมให้เป็นทั่วไปยิ่งขึ้นได้ กล่าวคือ สำหรับประพจน์ $\mathbf{A}, \mathbf{B}$ ใด ๆ %
		\boxthis{%
			\fitch{		
				\have[m]{1}		{\mathbf{A}}%
				\have[n]{a}		{\mathbf{B}}%
				\have[~]{3}		{\mathbf{A} \wedge \mathbf{B}}	\ai{1,a}
			}
			}
		}
	
		การเขียนเช่นนี้ไม่ได้หมายความว่า $\mathbf{A}$ในลักษณะเดียวกันกับกฎการเติม เรายังมีกฎการตัดสำหรับตัวเชื่อมประพจน์ ``และ'' (Conjuction elimination) ซึ่งสามารถอธิบายได้ดังนี้
		
		สมมติว่าเราทราบแล้วว่ากรุงเทพเป็นเมืองหลวงของประเทศไทย และประเทศฝรั่งเศสอยู่ในทวีปยุโรป เราต้องสรุปได้ว่ากรุงเทพเป็นเมืองหลวงของประเทศไทย เช่นกันเราต้องสรุปได้ว่าประเทศฝรั่งเศสอยู่ในทวีปยุโรป ในระบบนิรนัยธรรมชาติกฎข้อนี้คือ
		
		\begin{textbox}
			\fitch{		
				\have[m]{1}		{\mathbf{A} \wedge \mathbf{B}}%
				\have[~]{3}		{\mathbf{A}}					\ae{1}
			}
		\end{textbox}
		และเราย่อมได้เช่นกันว่า
		
		\boxthis{
			\fitch{		
				\have[m]{1}		{\mathbf{A} \wedge \mathbf{B}}%
				\have[~]{3}		{\mathbf{B}}					\ae{1}
			}
		}
		สัญลักษณ์ $\wedge\text{E}$ มาจากคำว่า Conjuction Elimination นั่นเอง
		
		การอ้างบรรทัดสำหรับกฎ $\wedge$I สามารถอ้างบรรทัดเดิมได้ การพิสูจน์ด้านล่าง
		\fitch{
			\hypo[1]{1}		{Q}
			\have[2]{2}		{Q \wedge Q}	\ai{1,1}
			\have[3]{3}		{Q}				\ae{1}
		}
		เป็นการพิสูจน์ที่สมเหตุสมผล
		
		\section{ตัวเชื่อมประพจน์ \q{ถ้า-แล้ว}}
		ในส่วนนี้เราจะแนะนำกฎการเติมและกฎการตัดของตัวเชื่อมประพจน์ \q{ถ้า-แล้ว} หรือประพจน์แบบเงื่อนไข (Conditional) ซึ่งเป็นข้อความที่ประกอบด้วยประพจน์สองส่วนคือส่วนเหตุ (Antecedent) และส่วนผล (Consequent) 
		
		ลองพิจารณาตัวอย่างด้านล่าง
		
		\begin{center}
			ถ้าตอนนี้ฝนตกแล้วถนนจะเปียก ตอนนี้ฝนตก ดังนั้นถนนเปียก
		\end{center}
	
		ท่านผู้อ่านเห็นได้ไม่ยากว่าการอ้างเหตุผลข้างต้นเป็นจริง ข้อความด้านบนสามารถระบุสัญลักษณ์แทนประพจน์ได้ดังนี้
		\begin{itemize}
			\item[]	$R$: ตอนนี้ฝนตก
			\item[]	$W$: ถนนเปียก
		\end{itemize}
		จะเห็นว่าประพจน์แบบเงื่อนไขของการอ้างเหตุผลคือ `$R \to W$' เราเรียก `$R$' ว่าส่วนเหตุ และ `$W$' ว่าส่วนผล เราสามารถตั้งกฎการตัดสำหรับตัวเชื่อมประพจน์ \q{ถ้า-แล้ว} ได้จากการอ้างเหตุผลข้างต้น
		\boxthis{
			\fitch{
			\have[m]{1}	{\mathbf{A} \to \mathbf{B}}
			\have[n]{2}	{\mathbf{A}}
			\have[~]{3}	{\mathbf{B}}	\ie{1,2}
			}	
		}
		กฎข้างต้นมีอีกชื่อว่า \textit{Modus ponens} ซึ่งเป็นการอ้างเหตุผลที่สำคัญแบบหนึ่ง  
		
		ลำดับของประพจน์ในการพิสูจน์ไม่สำคัญ ตัวอย่างเช่น
		\fitch{
			\have[2]{1}	{R \vee S}
			\have[8]{2}	{(R \vee S) \to Q}
			\have[9]{3}	{Q}	\ie{2,1}
		}
		การพิสูจน์ข้างต้นยังสมเหตุสมผลอยู่ สังเกตลำดับการเขียนเลขอ้างอิงในบรรทัดสุดท้าย เราจะให้ลำดับเลขอ้างอิงของประพจน์แบบเงื่อนไขไว้ด้านหน้าเสมอ แล้วจึงตามด้วยเลขอ้างอิงของส่วนเหตุของประพจน์แบบมีเงื่อนไข
		
		สำหรับกฎการเติมสำหรับตัวเชื่อมประพจน์ \q{ถ้า-แล้ว} จะแตกต่างไปสักเล็กน้อย สมมติผู้อ่านต้องการพิสูจน์ว่า
		\begin{center}
			\begin{minipage}{0.6\textwidth}
				ถ้าตอนนี้ฝนตกและตอนนี้เป็นเวลาเที่ยงตรง แล้วฉันจะอยู่บ้าน 
				
				ตอนนี้เป็นเวลาเที่ยงตรง	
				
				ดังนั้นถ้าตอนนี้ฝนตก ฉันจะอยู่บ้าน
			\end{minipage}
		\end{center}
	
		อาศัยการให้สัญลักษณ์ด้านล่าง
		\begin{itemize}
			\item []	$R$: ตอนนี้ฝนตก
			\item []	$T$: ตอนนี้เป็นเวลาเที่ยงตรง
			\item []	$H$: ฉันอยู่บ้าน
		\end{itemize}
	
		การอ้างเหตุผลข้างต้นเขียนได้เป็น
		\begin{equation*}
			(R\wedge T) \to H, T \therefore	R \to H
		\end{equation*}
		ข้อตั้งทั้งหมดสามารถเขียนในการพิสูจน์ของเราได้เป็น
		\fitch{
			\hypo[1]{1}		{(R \wedge T) \to H}
			\hypo[2]{2}		{T}
		}
	 	ผู้อ่านสามารถตรวจการให้เหตุผลข้างต้นว่าเป็นจริงได้โดยใช้ตารางค่าความจริง แต่วิธีพิสูจน์ที่ง่ายและตรงไปตรงมาคือ ผู้อ่านสมมติ\textit{เพิ่มเติม}ว่าฝนตก แล้วพิสูจน์ให้ได้ว่า ฉันอยู่บ้าน โดยอาศัยประพจน์อื่น ๆ และประพจน์ที่สมมติเพิ่มมาอีก ในระบบการพิสูจน์ของเราสามารถเขียนเป็นสัญลักษณ์ได้ว่า
		\fitch{
			\hypo[1]{1}		{(R \wedge T) \to H}
			\hypo[2]{2}		{T}
			\open
			\hypo[3]{3}		{R}
			\close			
		}
		บรรทัดที่ 3 \textbf{ไม่ได้}เป็นผลของการให้เหตุผลจากบรรทัดอื่น ๆ การเขียนเว้นย่อหน้าบ่งบอกว่าบรรทัดนี้เป็นข้อตั้งเพิ่มเติมที่สมมติมาเพิ่มเพื่อใช้ในการพิสูจน์ ซึ่งสามารถดำเนินการต่อได้ดังนี้
		\fitch{
			\hypo[1]{1}		{(R \wedge T) \to H}
			\hypo[2]{2}		{T}
			\open
			\hypo[3]{3}		{R}
			\have[4]{4}		{R \wedge T}	\ai{3,2}
			\have[5]{5}		{H}				\ie{1,4}
			\close
		}
		ขอให้ผู้อ่านพิจารณาการอ้างอิง $\wedge\text{I}$ และ $\to$E ในการพิสูจน์ข้างต้นให้ถี่ถ้วน	และเนื่องจากเราแสดงให้เห็นแล้วว่า หากเราสมมติเพิ่มว่า `$R$' เราสามารถได้ว่า `$H$' หรืออีกนัยหนึ่ง ถ้าเกิด `$R$' แล้วย่อมเกิด `$H$'  ทำให้เราสรุปได้ว่า `$R \to H$' ดังนั้นเราจะได้ว่า
		\fitch{
			\hypo[1]{1}		{(R \wedge T) \to H}
			\hypo[2]{2}		{T}
			\open
			\hypo[3]{3}		{R}
			\have[4]{4}		{R \wedge T}	\ai{3,2}
			\have[5]{5}		{H}				\ie{1,4}
			\close
			\have[6]{6}		{R \to H}		\ii{3-5}
		}
		สังเกตว่าเราเลิกเว้นย่อหน้าและกลับเข้ามาเขียนโดยใช้เส้นตั้งเส้นเดียว นั่นคือเราได้\textit{สละ}ข้อตั้งเพิ่มจากการพิสูจน์แล้ว บรรทัดอื่น ๆ ถัดจากนี้จะไม่สามารถอ้างบรรทัดที่ 3-5 ซึ่งมาจากการสมมติว่า `$R$'
		
		โดยทั่วไปแล้ว กฎการเติมสำหรับตัวเชื่อมประพจน์ \q{ถ้า-แล้ว} คือ
		\boxthis{
			\fitch{
			\open
				\hypo[m]{1}	{\mathbf{A}}
				\have[n]{2}	{\mathbf{B}}
			\close
			\have[~]{3}	{\mathbf{A} \to \mathbf{B}}		\ii{1-2}
			}
		}
		อีกตัวอย่างหนึ่ง ซึ่งเป็นการพิสูจน์การอ้างเหตุผลต่อไปนี้
		\begin{equation*}
			P \to Q, Q \to R \therefore P \to R
		\end{equation*}
		เริ่มต้นด้วยการเขียนข้อตั้งทั้งหมด
		\fitch{
			\hypo[1]{1}		{P \to Q}
			\hypo[2]{2}		{Q \to R}
		}
		เนื่องจากข้อสรุปของการอ้างเหตุผลคือ `$P \to R$' ดังนั้นเราจึงสมมติเพิ่มเติมว่า `$P$' แล้วจะพิสูจน์ให้ได้ว่า `$R$' เพื่อใช้กฎ $\to$I ในภายหลัง
		\fitch{
			\hypo[1]{1}		{P \to Q}
			\hypo[2]{2}		{Q \to R}
			\open
			\hypo[3]{3}		{P}
			\close
		}
		จากนั้นเราใช้กฎ $\to$E กับข้อตั้งทั้งหมดได้ ซึ่งทำให้ในตอนสุดท้ายเราจะได้ว่า `$R$' ซึ่งทำให้เราสรุปได้ว่า `$P\to R$'
				\fitch{
			\hypo[1]{1}		{P \to Q}
			\hypo[2]{2}		{Q \to R}
			\open
			\hypo[3]{3}		{P}
			\have[4]{4}		{Q}		\ie{1,3}
			\have[5]{5}		{R}		\ie{2,4}
			\close
			\have[6]{6}		{P \to R}	\ii{3-5}
		}
		\section{การพิสูจน์ย่อย}
		ดังที่ได้กล่าวไปข้างต้น เราไม่สามารถอ้างบรรทัดที่เกิดจากการสมมติได้ ลองพิจารณาตัวอย่างด้านล่าง
		\fitch{
			\hypo[1]{1}		{A}
			\open
			\hypo[2]{2}		{B}
			\have[3]{3}		{B \wedge B}	\ai{2,2}
			\have[4]{4}		{B}				\ae{3}
			\close
			\have[5]{5}		{B \to B}	\ii{2-4}
		}
		การอ้างเหตุผลด้านบนเป็นการอ้างเหตุผลที่สมเหตุสมผล ประพจน์ `$B \to B$' เป็นสัจนิรันดร์ในตัวของมันเองจึงไม่เกิดปัญหาอะไร แต่ถ้าเราจะพยายามอ้างเหตุผลจากบรรทัดของการสมมติ
		\fitch{
				\hypo[1]{1}		{A}
				\open
				\hypo[2]{2}		{B}
				\have[3]{3}		{B \wedge B}	\ai{2,2}
				\have[4]{4}		{B}				\ae{3}
				\close
				\have[5]{5}		{B \to B}	\ii{2-4}
				\have[6]{6}		{B}			\by{$\to$E?}{5,4}
		}
		การพิสูจน์ข้างต้นมีปัญหาแน่นอน เพราะเราพิสูจน์ได้ว่าประพจน์ทุกประพจน์เป็นจริง! ถ้าหากฝนตก (แทนด้วย `$A$') เราไม่ควรสรุปได้ว่าแมวเป็นสัตว์ปีก (แทนด้วย `$B$') เด็ดขาด ดังนั้นการอ้างเหตุผลข้างต้นต้องทำไม่ได้ จึงเป็นที่มาของแนวคิดเรื่อง\textit{การพิสูจน์ย่อย} (Subproof) 
		
		การพิสูจน์ถ้ามีการอ้างข้อตั้งเพิ่ม (เหมือนในตัวอย่างข้างต้น) เราเรียกว่าเรากำลัง\textit{พิสูจน์ย่อย} ซึ่งแสดงให้เห็นด้วยการเขียนเส้นกั้นแนวตั้งเพิ่มเติม และเว้นย่อหน้าเพิ่มขณะพิสูจน์ย่อย เมื่อเสร็จการพิสูจน์ย่อยเราจะ\textit{สละ}ข้อตั้งเพิ่มเติม และเป็นการปิดการพิสูจน์ย่อย \textbf{บรรทัดอื่น ๆ หลังปิดการพิสูจน์ย่อย จะอ้างผลจากบทพิสูจน์ย่อยไม่ได้เด็ดขาด}
		
		
		ในตัวอย่างข้างต้น บรรทัดที่ 2 เริ่มการพิสูจน์ย่อย และบรรทัดที่ 4 เป็นบรรทัดที่ปิดการพิสูจน์ย่อย ฉะนั้นบรรทัดที่ตามมาคือบรรทัดที่ 6 จะอ้างผลจากบรรทัดในบทพิสูจน์ย่อยซึ่งเริ่มจากบรรทัดที่ 2 ไปบรรทัดที่ 4 ไม่ได้ จึงทำให้การพิสูจน์ไม่สมเหตุสมผล
		
		แล้วเราทำการพิสูจน์ย่อยซ้อนกันไปได้อีกเรื่อย ๆ หรือไม่ คำตอบคือทำได้และแสดงในตัวอย่างถัดไป
		\fitch{
			\hypo[1]{1}		{A}
			\open
			\hypo[2]{2}		{B}
			\open
			\hypo[3]{3}		{C}
			\have[4]{4}		{A \wedge B}	\ai{1,2}
			\close
			\have[5]{5}		{C \to (A \wedge B)}	\ii{3-4}
			\close
			\have[6]{6}		{B \to (C \to (A \wedge B))}	\ii{2-5}
		}
		แต่การพิสูจน์ด้านล่างไม่ถูกต้อง (ผู้อ่านพอจะยกตัวอย่างประพจน์ได้หรือไม่)
		\fitch{
			\hypo[1]{1}		{A}
			\open
			\hypo[2]{2}		{B}
			\open
			\hypo[3]{3}		{C}
			\have[4]{4}		{A \wedge B}	\ai{1,2}
			\close
			\have[5]{5}		{C \to (A \wedge B)}	\ii{3-4}
			\close
			\have[6]{6}		{B \to (C \to (A \wedge B))}	\ii{2-5}
			\have[7]{7}		{C \to (A \wedge B)} 	\by{$\to$I?}{3-4}
		}
		ปัญหาที่เกิดขึ้นคือ บทพิสูจน์ย่อยที่เริ่มจากข้อตั้งเพิ่มว่า `$C$' นั้นอยู่ในบทพิสูจน์ย่อยที่เริ่มจากข้อตั้งเพิ่ม `$B$' อีกที และเมื่อถึงบรรทัดที่ 7 บทพิสูจน์ย่อยที่เริ่มจากข้อตั้งเพิ่ม `$B$' นั้นปิดแล้ว ผู้อ่านจะเห็นว่าบรรทัดที่ 5 เป็นการใช้กฎ $\to$I \textit{โดยมีข้อตั้งเพิ่มว่า `$B$'} ดังนั้นเมื่อเราสละข้อตั้ง `$B$' แล้ว เราควรไม่สามารถใช้กฎ $\to$I อย่างในบรรทัดที่ 5 ได้อีก ซึ่งสามารถสรุปเป็นกฎได้ว่า \textbf{การใช้กฎกับบทพิสูจน์ย่อย บทพิสูจน์ย่อยนั้นต้องไม่อยู่ในบทพิสูจน์ย่อยอื่นที่ปิดไปแล้ว}
		
		\section{ตัวเชื่อมประพจน์ \q{ก็ต่อเมื่อ}}
		เนื่องจากตัวเชื่อมประพจน์ \q{ก็ต่อเมื่อ} (Biconditional) มีลักษณะเป็นกับตัวเชื่อมประพจน์ \q{ถ้า-แล้ว} สองทาง การพิสูจน์นั้นทำคล้ายกับการพิสูจน์ \q{ถ้า-แล้ว} สองครั้ง หากจะพิสูจน์ข้อความที่ว่า `$A \leftrightarrow B$' เราจะสมมติให้ `$A$' แล้วพิสูจน์ว่า `$B$' จากนั้นเราจะสมมติว่า `$B$' แล้วแสดงให้ได้ว่า `$A$' ซึ่งสามารถตั้งกฎได้เป็น
		\boxthis{\fitch{
			\open
			\hypo[i]{i}		{\mathbf A}
			\have[j]{j}		{\mathbf B}
			\close
			\open
			\hypo[m]{m}		{\mathbf B}
			\have[n]{n}		{\mathbf A}
			\close
			\have[~]{1}		{\mathbf A \leftrightarrow \mathbf B}	\bii{i-j,m-n}
			}
		}
		และเนื่องจากตัวเชื่อมประพจน์ \q{ก็ต่อเมื่อ} คือตัวเชื่อมประพจน์ \q{ถ้า-แล้ว} สองทาง ดังนั้นเราจะได้ว่า
		\boxthis{\fitch{
				\have[i]{i}		{\mathbf A \leftrightarrow \mathbf B}
				\have[j]{j}		{\mathbf A}
				\have[~]{k}		{\mathbf B}	\bie{i,j}
			}
		}
		และเช่นกันว่า
		\boxthis{\fitch{
			\have[i]{i}		{\mathbf A \leftrightarrow \mathbf B}
			\have[j]{j}		{\mathbf B}
			\have[~]{k}		{\mathbf A}	\bie{i,j}
			}
		}
		สังเกตลำดับเลขอ้างอิง เราจะให้ประพจน์ \q{ก็ต่อเมื่อ} ขึ้นก่อนเสมอ
		
		\section{ตัวเชื่อมประพจน์ \q{หรือ}}
		สมมติเราทราบว่าตอนนี้ฝนตก เราสามารถกล่าวได้ว่า ตอนนี้ฝนตก\textit{หรือ}ถนนเปียก การพูดเช่นนั้นยังคงจริงอยู่ (เพราะแน่นอนว่าตอนนี้ฝนตก) ลักษณะข้างต้นเป็นลักษณะของข้อความที่เชื่อมด้วยตัวเชื่อมประพจน์ \q{หรือ} (Disjunction) ซึ่งทำให้ข้อความนั่นดูอ่อนลง
		
		และการพูดว่า ตอนนี้ฝนตก\textit{หรือ}ฝนไม่ตกยังคงจริง แม้กระทั่งหากจะกล่าวว่า ตอนนี้ฝนตก\textit{หรือ}พระเจ้ามีจริง ก็ยังคงจริงอยู่เช่นเดียวกัน ทั้งหมดนี้ย่อมเป็นผลจากข้อความที่ว่า ตอนนี้ฝนตก (แม้จะขัดกับการใช้ภาษาในชีวิตประจำวันก็ตาม)
		
		เราสามารถสร้างกฎการเติมสำหรับตัวเชื่อมประพจน์ \q{หรือ} (Disjunction introduction) ได้ว่า
		\boxthis{\fitch{
				\have[i]{i}		{\mathbf A }
				\have[~]{k}		{\mathbf A \vee \mathbf B}	\oi{i}
			}
		}
		ในทำนองเดียวกัน
		\boxthis{\fitch{
				\have[i]{i}		{\mathbf A }
				\have[~]{k}		{\mathbf B \vee \mathbf A}	\oi{i}
			}
		}
	
		เพื่อยกตัวอย่าง การอ้างเหตุผลด้านล่างนี้เป็นไปตามกฎการเติมสำหรับตัวเชื่อมประพจน์ \q{หรือ}
		\fitch{
			\hypo[1]{1}	{M}
			\have[2]{2}	{M \vee ([(A \leftrightarrow B) \to (C \wedge D)] \to (R \vee S))} \oi{1}
		}
		หากจะแสดงว่าข้อความนี้เป็นจริงโดยใช้ตารางความจริง จะมีบรรทัดทั้งหมด $2^7 = 128$ บรรทัด!
		
		กฎการตัดสำหรับตัวเชื่อมประพจน์ \q{หรือ} จะต่างไปสักเล็กน้อย สมมติให้ตอนนี้ฝนตกหรือตอนนี้แดดออก เราไม่อาจรู้ได้อย่างแน่นอนว่าข้อความไหนเป็นจริง เพราะไม่จำเป็นที่ตอนนี้ฝนตก เพราะตอนนี้แดดอาจจะออกก็ได้ และไม่จำเป็นว่าตอนนี้แดดออก เพราะตอนนี้ฝนอาจจะตกก็ได้ อย่างไรข้อความที่ว่า ตอนนี้ฝนตกหรือตอนนี้แดดออก ก็ยังคงเป็นจริง
		
		แต่หากเราสามารถแสดงได้ว่า ถ้าหากตอนนี้ฝนตก แล้วฉันไม่อยู่บ้าน และในขณะเดียวกัน ถ้าหากตอนนี้แดดออก แล้วฉันไม่อยู่บ้านเช่นกัน ไม่ว่ากรณีไหนก็ตาม เราต้องสรุปได้ว่า ตอนนี้ฉันไม่อยู่บ้าน ซึ่งทำให้เราได้กฎการตัดสำหรับตัวเชื่อมประพจน์ \q{หรือ} ตามด้านล่าง
		\boxthis{\fitch{
				\have[a]{a}		{\mathbf A \vee \mathbf B}
				\open
				\hypo[i]{i}		{\mathbf A}
				\have[j]{j}		{\mathbf C}
				\close
				\open
				\hypo[m]{m}		{\mathbf B}
				\have[n]{n}		{\mathbf C}
				\close
				\have[~]{1}		{\mathbf C}	\oe{a,i-j,m-n}
			}
		}
	
		ตัวอย่างการพิสูจน์โดยใช้กฎการตัด เพื่อพิสูจน์ว่า
		\begin{equation*}
			(P\wedge Q) \vee (R \wedge P) \therefore P
		\end{equation*}
		\fitch{
			\hypo[1]{1}		{(P\wedge Q) \vee (R \wedge P)}
			\open
			\hypo[2]{2}		{P\wedge Q}
			\have[3]{3}		{P}			\ae{2}
			\close
			\open
			\hypo[4]{4}		{R \wedge P}
			\have[5]{5}		{P}			\ae{4}
			\close
			\have[6]{6}		{P}			\oe{1,2-3,4-5}
		}
	
		อีกตัวอย่างหนึ่งซึ่งยากกว่าสักเล็กน้อย
		\begin{equation*}
			P\wedge(Q \vee R) \therefore (P\wedge Q) \vee (P \wedge R)
		\end{equation*}
		\fitch{
			\hypo[1]{1}		{P\wedge(Q \vee R)}
			\have[2]{2}		{P}				\ae{1}
			\have[3]{3}		{Q \vee R}	\ae{1}
			\open
			\hypo[4]{4}		{Q}
			\have[5]{5}		{P \wedge Q}	\ai{2,4}
			\have[6]{6}		{(P\wedge Q) \vee (P \wedge R)}		\oi{5}
			\close
			\open
			\hypo[7]{7}		{R}
			\have[8]{8}		{P \wedge R}	\ai{2,7}
			\have[9]{9}		{(P\wedge Q) \vee (P \wedge R)}		\oi{8}
			\close
			\have[10]{10}	{(P\wedge Q) \vee (P \wedge R)}		\oe{3,4-6,7-9}
		}
	\section{ตัวเชื่อมประพจน์ \q{ไม่} และข้อขัดแย้ง}
		ในการนิยามกฎการตัดและกฎการเติมสำหรับตัวเชื่อมประพจน์\footnote{ในที่นี้จะถือว่าสัญลักษณ์ $\neg$ เป็นตัวเชื่อมประพจน์ตัวหนึ่ง แม้จะไม่ได้เชื่อมประพจน์สองประพจน์เข้าด้วยกัน ทั้งนี้เพื่อให้เป็นไปตามศัพท์ภาษาอังกฤษว่า connective} \q{ไม่} จะต้องอาศัยแนวคิดเพิ่มเติมสักเล็กน้อย นั่นคือแนวคิด \textit{ข้อขัดแย้ง} (Contradiction) ข้อขัดแย้งเกิดขึ้นเมื่อข้อความที่เป็นนิเสธกันปรากฎขึ้นมาทั้งคู่ในการอ้างเหตุผล ในระบบการพิสูจน์ของเราจะใช้สัญลักษณ์ $\bot$ เพื่อบ่งบอกว่าเกิดข้อขัดแย้งในการพิสูจน์
		\boxthis{\fitch{
			\have[m]{1}		{\mathbf A}
			\have[n]{2}		{\neg \mathbf A}
			\have[~]{3}		{\bot}		\ne{1,2}
		}}
		ลำดับเลขอ้างอิงเริ่มต้นด้วยประพจน์นั้นก่อน แล้วจึงตามด้วยนิเสธของประพจน์นั้น ๆ กฎข้อนี้เป็นกฎการตัด เพราะเรากำจัดตัวเชื่อมประพจน์ \q{ไม่} ออกไป (จะถือว่านี่คือกฎ $\bot$I ก็ย่อมได้)
		
		และกฎการเติมสำหรับตัวเชื่อมประพจน์ \q{ไม่} นั้นง่ายมาก หากข้อตั้งเพิ่มหรือข้อสมมติทำให้เกิดข้อขัดแย้ง ข้อสมมตินั้นต้องไม่จริง จึงได้กฎตามมาว่า
		\boxthis{
			\fitch{
				\open
				\hypo[m]{1}		{\mathbf A}
				\have[n]{2}		{\bot}
				\close
				\have[~]{3}		{\neg \mathbf A}		\ni{1-2}
		}}
		ตัวอย่างด้านล่างเป็นการใช้กฎทั้งสองในการพิสูจน์ว่า $J \therefore \neg\neg J$
			\fitch{
			\hypo[1]{1}		{J}
			\open
			\hypo[2]{2}		{\neg J}
			\have[3]{3}		{\bot}		\ne{1,2}
			\close
			\have[4]{4}		{\neg\neg J}	\ni{2-3}	
		}
		ในเมื่อเราเพิ่มสัญลักษณ์ $\bot$ เข้ามาในระบบของเรา เราจะต้องอาศัยกฎเพื่อตัดสัญลักษณ์นี้ออก ซึ่งกฎที่เราจะใช้เรียกว่า \textit{ex falso quodlibet} ซึ่งหมายความว่า \textit{ทุกอย่างเป็นผลจากข้อขัดแย้ง} เมื่อไรก็ตามที่สัญลักษณ์นี้ปรากฎ เราสามารถอ้างเหตุผลทุกอย่างจากผลนี้ นี่เป็นเหตุผลที่ทำให้กฎข้อนี้มีอีกชื่อว่า \textit{principle of explosion} หรือ \textit{หลักการระเบิด}
		\boxthis{\fitch{
			\have[m]{m}		{\bot}
			\have[~]{1}		{\mathbf A}		\X{m}
		}}
		 สัญลักษณ์ X มาจากคำว่า \textit{ex falso quodlibet} หรือ eXplosion นั่นเอง
	\section{กฎนิรมัชฌิม}
		สำหรับกฎสุดท้ายที่จะใช้ในระบบการพิสูจน์เป็นกฎที่เกี่ยวข้อกับตัวเชื่อมประพจน์ \q{ไม่} เช่นกัน สมมติว่า หากตอนนี้แดดออก สมหญิงจะพกร่ม (กันแดดเผา) และหากตอนนี้แดดไม่ออก สมหญิงก็จะพกร่ม (กันฝนตก) ไม่ว่าอย่างไรสมหญิงก็จะพกร่มอยู่เสมอ ไม่ว่าอย่างไรก็ตาม
		
		กฎดังกล่าวจึงเขียนในระบบพิสูจน์ได้เป็น
		\boxthis{\fitch{
				\open
				\hypo[i]{i}		{\mathbf A}
				\have[j]{j}		{\mathbf B}
				\close
				\open
				\hypo[m]{m}		{\neg \mathbf A}
				\have[n]{n}		{\mathbf B}
				\close
				\have[~]{1}		{\mathbf B}	\TND{i-j,m-n}
			}
		}
		กฎข้อนี้มีชื่อว่า \textit{tertium non datur} ซึ่งแปลว่า \textit{ไม่มีทางที่สาม} ตัวอย่างต่อไปนี้เป็นการใช้กฎข้อนี้เพื่อพิสูจน์ว่า
		\begin{equation*}
			D \to (A \wedge R), \neg D \to R \therefore R
		\end{equation*}
		\fitch{
			\hypo[1]{1}		{D \to (A \wedge R)}
			\hypo[2]{2}		{\neg D \to R}
			\open
			\hypo[3]{3}		{D}
			\have[4]{4}		{A \wedge R}	\ie{1,3}
			\have[5]{5}		{R}				\ae{4}
			\close
			\open
			\hypo[6]{6}		{\neg D}
			\have[7]{7}		{R}			\ie{2,6}
			\close
			\have[8]{8}		{R}		\TND{3-5,6-7}
		}
		ทั้งหมดนี้คือกฎพื้นฐานของระบบการพิสูจน์สำหรับ TFL
		\section{แบบฝึกหัด}
		\begin{enumerate}
			\item จงหาข้อผิดพลาดของการพิสูจน์ด้านล่าง
			\begin{multicols}{2}
				\fitch{
					\hypo[1]{1}		{P\vee Q}
					\hypo[2]{2}		{Q}
					\open
					\hypo[3]{3}		{P}
					\have[4]{4}		{P \wedge Q}	\ai{4,2}
					\close
					\open
					\hypo[5]{5}		{Q}
					\have[6]{6}		{P\wedge Q}		\ai{3,5}
					\close
					\have[7]{7}		{P \wedge Q}	\oe{1,3-4,5-6}
				}
				\fitch{
				\hypo[1]{1}		{(A \wedge D) \wedge F}
				\hypo[2]{2}		{A \to G}
				\have[3]{3}		{A}			\ae{1}
				\have[4]{4}		{G}		\ie{2,3}
			}
			\end{multicols}
		
			\item จงเติมการอ้างอิงสำหรับบทพิสูจน์ต่อไปนี้ให้สมบูรณ์
			
			\begin{multicols}{2}
				\fitch{
					\hypo[1]{1}		{A \wedge G}
					\hypo[2]{2}		{G \to H}
					\hypo[3]{3}		{A \to B}
					\have[4]{4}		{A}
					\have[5]{5}		{B}
					\have[6]{6}		{G}
					\have[7]{7}		{H}
					\have[8]{8}		{H \wedge B}
				}
			
				\fitch{
					\hypo[1]{1}		{\neg A \to (B \wedge C)}
					\hypo[2]{2}		{(B \wedge E)\to D}
					\hypo[3]{3}		{C \wedge E}
					\have[4]{4}		{E}
					\open
					\hypo[5]{5}		{\neg A}
					\have[6]{6}		{B \wedge C}
					\have[7]{7}		{B}
					\have[8]{8}		{B \wedge E}
					\have[9]{9}		{D}
					\have[10]{10}	{D \vee E}
					\close
					\have[11]{11}	{\neg A \to (D \vee E)}
				}		
			\end{multicols}
			\begin{multicols}{2}
								\fitch{
					\hypo[1]{1}		{R \to P}
					\open
					\hypo[2]{2}		{R}
					\have[3]{3}		{P}
					\have[4]{4}		{P \vee Q}
					\have[5]{5}		{(P\vee Q)\wedge R}
					\close
					\have[6]{6}		{R \to [(P\vee Q) \wedge R]}
				}
				\fitch{
					\hypo[1]{3}		{(F \vee K) \to H}
					\hypo[2]{3}		{J \vee F}
					\hypo[3]{4}		{\neg J}
					\open
					\hypo[4]{5}		{J}
					\have[5]{6}		{\bot}
					\have[6]{7}		{H}
					\close
					\open
					\hypo[7]{8}		{F}
					\have[8]{9}		{F \vee K}
					\have[9]{10}	{H}
					\close
					\have[10]{13}	{H}
				}
			\end{multicols}
		
			\item จงพิสูจน์การอ้างเหตุผลต่อไปนี้ว่าเป็นสมเหตุสมผล
			
			\begin{enumerate}[label={\arabic*.}]
				\item $J \to \neg J \therefore \neg J$
				\item $A \to (B \to C) \therefore (A \wedge B )\to C$
				\item $\neg A \therefore A \to B$
				\item $\neg (A \wedge B) \therefore \neg A \vee \neg B$
				\item $A \to (B \to C) \therefore ([A \to B] \to [A \to C])$
				\item $\neg F \to G, F \to H \therefore G \vee H$
				\item $S \leftrightarrow T \therefore S \leftrightarrow (T \vee S)$
				\item $D \to (A \vee \neg B), D \to \neg A, (\neg B \vee C) \to (G \wedge \neg D) \therefore \neg D$
				\item $ C \to A, A \to (A \to B), \neg D \to (C \to B) \therefore B \vee D$
				\item $\neg(P \to Q) \therefore \neg Q$
				\item $\neg(P\to Q) \therefore P$
			\end{enumerate}
		\end{enumerate}
	
	\chapter{กฎเพิ่มเติมสำหรับตรรกศาสตร์เชิงฟังก์ชันความจริง}
		ในบทที่แล้วเราได้เสนอกฎสำหรับการนิรนัยธรรมชาติไปแล้ว เราจะเสนอกฎเพิ่มเติมสำหรับระบบของเรา กฎทั้งหมดต่อไปนี้สามารถพิสูจน์ได้จากกฎพื้นฐานทั้งหมด แต่จะทำให้เราทำงานในระบบการพิสูจน์ได้ง่ายขึ้น
		\section{กฎการเขียนซ้ำ}
		กฎการเขียนซ้ำ (Reiteration) ทำให้เราสามารถเขียนประพจน์ใด ๆ ได้อีกครั้งหนึ่ง
		\boxthis{\fitch{
			\have[m]{m}		{\mathbf A}
			\have[~]{1}		{\mathbf	A}	\r{m}
			}
		}
		ตัวอย่างประโยชน์ของกฎข้อนี้
		\fitch{
			\hypo[1]{1}		{P \vee Q}
			\hypo[2]{2}		{P \to Q}
			\open
			\hypo[3]{3}		{P}
			\have[4]{4}		{Q}		\ie{2,3}
			\close
			\open
			\hypo[5]{5}		{Q}
			\have[6]{6}		{Q}		\r{5}
			\close
			\have[7]{7}		{Q}		\oe{1,3-4,5-6}
		}
		ผู้อ่านสามารถลองพิสูจน์การพิสูจน์ข้างต้นโดยใช้เพียงกฎพื้นฐานเท่านั้น
		\section{ตรรกบทแบบการเลือก}
		กฎนี้เป็นกฎที่เห็นได้ชัดกฎหนึ่ง ลองดูตัวอย่าง
		\begin{center}
			ฉันพกร่มหรือฉันพกแว่นกันแดด ฉันไม่ได้พกแว่นกันแดด ดังนั้นฉันพกร่ม
		\end{center}
		
		การให้เหตุผลข้างต้นเรียกว่าตรรกบทแบบการเลือก (Disjunctive syllogism) ซึ่งมีรูปแบบดังนี้
		\boxthis{\fitch{
			\have[m]{m}		{\mathbf A \vee\mathbf B}
			\have[n]{n}		{\neg\mathbf A}
			\have[~]{1}		{\mathbf B}		\DS{m,n}
		}}
		และ
		\boxthis{\fitch{
				\have[m]{m}		{\mathbf A \vee\mathbf B}
				\have[n]{n}		{\neg\mathbf B}
				\have[~]{1}		{\mathbf A}		\DS{m,n}
		}}
		\section{Modus tollens}
			พิจารณาตัวอย่างด้านล่าง
		\begin{center}
			ถ้าฉันกินแค่ขนมปังเป็นอาหารเช้า	ตอนเที่ยงฉันจะหิว	ตอนเที่ยงฉันไม่หิว ดังนั้นฉันไม่ได้กินแค่ขนมปังเป็นอาหารเช้า
		\end{center}
		รูปแบบการอ้างเหตุผลนี้เรียกว่า \textit{modus tollens} ซึ่งเป็นไปตามกฎดังนี้
		\boxthis{\fitch{
				\have[m]{m}		{\mathbf A \to\mathbf B}
				\have[n]{n}		{\neg\mathbf B}
				\have[~]{1}		{\neg\mathbf A}		\MT{m,n}
		}}
		เราจะอ้างประโยคแบบเงื่อนไขก่อนเสมอ
		\section{กฎการตัดนิเสธซ้อน}
		เราสามารถตัดนิเสธซ้อนกันสองครั้ง (double negation elimination) ได้ ตามรูปแบบด้านล่าง
		\boxthis{\fitch{
				\have[m]{m}		{\neg\neg\mathbf A }
				\have[~]{1}		{\mathbf A}		\nne{m}
		}}
		\section{กฎเดอมอร์แกน}
		กฎสุดท้ายที่เราจะแนะนำในระบบของเราคือกฎเดอมอร์แกน (De Morgan rules)\footnote{ตั้งชื่อเพื่อเป็นเกียรติให้แก่ Augustus De Morgan (1806-1871)} ซึ่งมีทั้งหมด 4 ข้อและเป็นบทกลับของกัน
		
		กฎเดอมอร์แกนข้อแรกมีว่า
		\boxthis{\fitch{
				\have[m]{m}		{\neg(\mathbf A \vee\mathbf B)}
				\have[~]{1}		{\neg\mathbf A \wedge \neg\mathbf B}		\dem{m}
		}}
	 	กฎข้อที่สองเป็นบทกลับของข้อแรก
		 \boxthis{\fitch{
		 		\have[m]{m}		{\neg\mathbf A \wedge \neg\mathbf B}
		 		\have[~]{1}		{\neg(\mathbf A \vee\mathbf B)}		\dem{m}
		 }}
	 	กฎข้อที่สามเป็น \textit{ทวิภาค} (dual) ของข้อแรก
		\boxthis{\fitch{
		\have[m]{m}		{\neg(\mathbf A \wedge\mathbf B)}
		\have[~]{1}		{\neg\mathbf A \vee \neg\mathbf B}		\dem{m}
		}}
		และกฎข้อสุดท้ายเป็นบทกลับของข้อที่สาม
		\boxthis{\fitch{
				\have[m]{m}		{\neg\mathbf A \vee \neg\mathbf B}
				\have[~]{1}		{\neg(\mathbf A \wedge\mathbf B)}		\dem{m}
		}}
	\chapter{การพิสูจน์ในระบบตรรกศาสตร์อันดับแรก}
	
	\chapter{แนวคิดเชิงทฤษฎีการพิสูจน์และอรรถศาสตร์}
	เราจะแนะนำสัญลักษณ์ใหม่โดยจะเขียนว่า
	\begin{equation*}
		\mathbf{A}_1, \mathbf{A}_2, \ldots, \mathbf{A}_n \vdash \mathbf{C}
	\end{equation*}
	เมื่อมีบทพิสูจน์ที่ได้ผลสรุปว่า $\mathbf{C}$ และมีข้อตั้งที่ไม่สละทิ้งคือ $\mathbf{A}_1, \mathbf{A}_2, \ldots, \mathbf{A}_n$ 
	
	ในทางกลับกันเราจะเขียนว่า
	\begin{equation*}
	\mathbf{A}_1, \mathbf{A}_2, \ldots, \mathbf{A}_n \nvdash \mathbf{C}
	\end{equation*}
	ถ้าไม่มีบทพิสูจน์ดังกล่าว
	
	สังเกตว่าสัญลักษณ์ที่ใช้ไม่ใช่สัญลักษณ์ ``$\vDash$''
	\part{การพิสูจน์เชิงคณิตศาสตร์}
		\chapter{แนวคิดพื้นฐานในคณิตศาสตร์}
		\section{เซต}
		ในที่นี้เราจะใช้ทฤษฎีที่เกี่ยวกับเซตที่เรียกว่าทฤษฎีเซตอย่างพื้น\footnote{ผู้เขียนเลือกใช้คำนี้แทนคำทับศัพท์ที่ปรากฏทั่วไปของคำว่า naive เพื่อให้เห็นชัดว่าเป็นทฤษฎีเซตอันไม่ได้ขัดเกลา จึงทำให้เกิดช่องโหว่ในรูปของปฏิทรรศน์ต่าง ๆ ซึ่งจะกล่าวถึงเพิ่มเติมในภายหลัง} (Naive set theory) ข้อดีคือเป็นทฤษฎีที่ง่ายต่อการเข้าใจ
		
		แต่คำถามคือ \textit{เซตคืออะไร ?} เกออร์ก คันตอร์ (Georg Cantor) ผู้ริเริ่มทฤษฎีเซต ได้ให้ความหมายของเซตไว้ว่า
		
		\begin{figure}[h]
			``เมื่อกล่าวถึง ``ส่วนสะสม''(aggregrate) [เซต] ให้เข้าใจร่วมกันว่าเป็นการรวบรวมบรรดาสิ่งต่าง ๆ อันระบุเจาะจงและแตกต่าง $m$ ในความคิดของเรา ให้เป็นส่วนร่วมเดียว $M$ สิ่งต่าง ๆ นั้นเรียกว่าเป็นสมาชิกของ $M$''
		\end{figure}
		
		
		ปัญหาที่ตามมาคือ คำว่า ``การรวบรวม'' หรือ ``สิ่งต่าง ๆ'' มีความหมายว่าอย่างไร แต่เราจะถือว่าเราเข้าใจโดยชัดแจ้งว่าเซตในความคิดของคันตอร์เป็นอย่างไร
		
		\begin{definition}
			ให้ $S$ เป็นเซต จะเขียนแทน $x \in S$ ว่าคือ $x$ เป็นสมาชิกของ $S$ ในทางกลับกัน $y \notin S$ แทนความหมายว่า $y$ ไม่เป็นสมาชิกของ $S$
		\end{definition}
		เซตที่เรามักจะเจอได้ทั่วไป จนมีสัญลักษณ์สากลแทนเซตนั้น ได้แก่
		\begin{itemize}
			\item $\mathbb{N}$ แทนเซตของจำนวนนับหรือจำนวนธรรมชาติ\footnote{ในหนังสือบางที่จะรวมศูนย์ไว้ในเซตของจำนวนนับ ซึ่งในที่นี้จะใช้สัญลักษณ์แยกเพื่อให้ไม่กำกวม} $\Set{1, 2, \ldots}$
			\item $\mathbb{N}_0$ แทนเซตของจำนวนนับหรือจำนวนธรรมชาติรวมศูนย์ $\Set{0, 1, 2, \ldots}$
			\item $\mathbb{Z}$ แทนเซตของจำนวนเต็ม
			\item $\mathbb{Q}$ แทนเซตของจำนวนตรรกยะ
			\item $\mathbb{R}$ แทนเซตของจำนวนจริง
			\item $\mathbb{C}$ แทนเซตของจำนวนเชิงซ้อน
		\end{itemize}
		ในหนังสือเล่มนี้ เราจะเริ่มต้นระบบของเราจากจำนวนนับ 
		
		\section{ทฤษฎีจำนวน}
		\section{ระบบจำนวนจริง}
		
		\chapter{การพิสูจน์ทางตรง}
		
		\chapter{การพิสูจน์ข้อขัดแย้ง}
		
		\chapter{การพิสูจน์แย้งสลับที่}
		
		\chapter{การพิสูจน์ข้อความ \q{หรือ}}
		
		\chapter{การพิสูจน์โดยการแบ่งกรณี}
	
		\chapter{การอุปนัยเชิงคณิตศาสตร์}
\end{document}